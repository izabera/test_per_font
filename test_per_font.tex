\documentclass[12pt,a4paper]{article}

\usepackage{nopageno}
\usepackage{fontspec}
\usepackage{microtype}
\usepackage{polyglossia}
\setmainlanguage{italian}

\usepackage{yfonts}%per fraktur
\usepackage{xcolor}
\usepackage{xifthen}
\definecolor{bittersweet}{rgb}{1.0, 0.44, 0.37}
\definecolor{firstblue}{RGB}{51, 148, 193}
\definecolor{kindared}{RGB}{241, 89, 48}
\definecolor{secondblue}{RGB}{91, 196, 191}
\definecolor{yellowish}{RGB}{255, 194, 33}
\definecolor{purplous}{RGB}{152, 124, 173}
\definecolor{greenthing}{RGB}{100, 168, 117}

\newcommand{\mydummy}{``Quaranta morti!''---esclam\`o Xao Jin. ``Un massacro epocale, \`e stata una Waterloo.''

\`E così che venne annunciata la sconfitta al boss della yakuza.

Yamaguchi non rispose. Lentamente si alz\`o in piedi e and\`o verso la finestra del suo ufficio. Zitto. Immobile.

Suo figlio Kaneda era morto in uno scontro come quello, ma lui era rimasto inflessibile come sempre. Dall'inizio della guerra erano stati uccisi cos\`i tanti uomini...

Avrebbero vinto? Certo nessuno poteva saperlo. Non erano i quaranta morti il problema, ma le dozzine di perch\'e che sollevavano, domande a cui non sapeva rispondere. Troppe battaglie da affrontare, troppo affilati i coltelli degli avversari, troppo letali i loro attacchi... Vent'anni prima sarebbe andato lui stesso l\`a a combattere, ma ora non ne aveva le forze.

Pens\`o alla sua Hyoko, vittima innocente della mano violenta dei Matowaji. Rimase l\`i per un po', mentre i ricordi affluivano. Già da anni non pensava pi\`u ad Oxford, all'incendio di Birmingham, alle sparatorie di Edimburgo del 1973--74... Forse ormai era troppo vecchio.}

\newcounter{colorcounter}

\newcommand{\mycolor}{\stepcounter{colorcounter}%
\ifthenelse{\equal{\value{colorcounter}}{1}}{\color{firstblue}}{}%
\ifthenelse{\equal{\value{colorcounter}}{2}}{\color{kindared}}{}%
\ifthenelse{\equal{\value{colorcounter}}{3}}{\color{secondblue}}{}%
\ifthenelse{\equal{\value{colorcounter}}{4}}{\color{yellowish}}{}%
\ifthenelse{\equal{\value{colorcounter}}{5}}{\color{purplous}}{}%
\ifthenelse{\equal{\value{colorcounter}}{6}}{\color{greenthing}}{}%
}

\newcommand{\resetcolor}{\ifthenelse{\equal{\value{colorcounter}}{6}}{\setcounter{colorcounter}{0}}{}}
\newcommand{\testfont}[4][]{\vspace*{\fill}%
{#2%
\centerline{\textbf{\huge #3}}\par\vspace*{2cm}%
\centerline{\scalebox{8}{\mycolor{}#4\resetcolor{}}}\par\vspace*{2cm}%
\addfontfeatures{Ligatures=TeX}\mydummy%
\ifthenelse{\equal{#1}{}}{}{\\\color{bittersweet}\centerline{#1}}%
}%
\vspace*{\fill}\newpage}
%esempio:\testfont[figo]\bembo{Bembo Std}

\newcommand{\parte}[2]{\vspace*{\fill}%
\hspace*{\fill}\textsf{\textbf{\Huge#1}}\hspace*{\fill}%
\vskip 2cm%
\textsf{\noindent\large#2}%
\vspace*{\fill}\newpage}

\newfontfamily\baskervald{Baskervald ADF Std}
\newfontfamily\bembo{Bembo Std}
\newfontfamily\scala{Scala}
\newfontfamily\clifford{CliffordNine LF}
\newfontfamily\caslon{Adobe Caslon Pro}
\newfontfamily\celeste{CelesteST-Regular}
\newfontfamily\garamond{Adobe Garamond Pro}
\newfontfamily\jenson{Adobe Jenson Pro}
\newfontfamily\libertine{Linux Libertine O}
\newfontfamily\minion{Minion Pro}
\newfontfamily\palatino{Palatino LT Std}
\newfontfamily\raleigh{Raleigh LT Std}
\newfontfamily\sabon{Sabon LT Std}
\newfontfamily\troman{Times New Roman}
\newfontfamily\bodoni{Bauer Bodoni Std 1}
\newfontfamily\century{New Century Schoolbook LT Std}
\newfontfamily\melior{Melior LT Std}
\newfontfamily\grotesque{Grotesque MT Std}
\newfontfamily\univers{Univers LT 55}
\newfontfamily\helvetica{Helvetica Neue}
\newfontfamily\arial{Arial}
\newfontfamily\bau{BauLF-Regular}
\newfontfamily\avantgarde{ITC Avant Garde Gothic Std Book}
\newfontfamily\din{DINPro}
\newfontfamily\futura{Futura LT Book}
\newfontfamily\gill{Gill Sans MT}
\newfontfamily\frutiger{Frutiger LT 55 Roman}
\newfontfamily\myriad{Myriad Pro}
\newfontfamily\verdana{Verdana}
\newfontfamily\optima{Optima LT}
\newfontfamily\dax{Dax Light}
\newfontfamily\meta{MetaPro}
\newfontfamily\balance{Balance-Regular}
\newfontfamily\rockwell{Rockwell}
\newfontfamily\caecilia{Caecilia LT Std}
\newfontfamily\bickham[Scale=1.4]{Bickham Script Pro}
\newfontfamily\zapfino[Scale=1.3]{LTZapfino One}
\newfontfamily\pagella{TeX Gyre Pagella}
\newfontfamily\termes{TeX Gyre Termes}
\newfontfamily\cursor[Scale=0.9]{TeX Gyre Cursor}
\newfontfamily\heros{TeX Gyre Heros}
\newfontfamily\chorus{TeX Gyre Chorus}
\newfontfamily\schola{TeX Gyre Schola}
\newfontfamily\bonum{TeX Gyre Bonum}
\newfontfamily\adventor{TeX Gyre Adventor}
\newfontfamily\crimson{Crimson Text}
\newfontfamily\ebgaramond{EB Garamond}

\begin{document}
\parte{Veneziani o umanisti}{Molto calligrafici, con un'angolatura evidente e moderato contrasto nel tratto. Spesso mostrano grazie asimmetriche.}
\testfont{\jenson}{Adobe Jenson}{RingrAziare}
\testfont{\minion}{Minion}{Dimentichi}
\testfont{\crimson}{Crimson}{Imbottiva}
\testfont[Versione open source di Palatino]{\pagella}{TeX Gyre Pagella}{LongoBardi}
%\testfont{\palatino}{Palatino LT Std}{Ciao}

\parte{Romani antichi}{Le grazie sono solitamente concave o piatte con terminali rotondi. C'è molta differenza tra aste orizzontali e verticali.}
\testfont{\bembo}{Bembo}{Xenofobia}
\testfont[Da risolvere: non mostra le legature]{\clifford}{FF Clifford}{VestiTo}
\testfont{\scala}{FF Scala}{YogurteRia}
\testfont[Versione open source di Adobe Garamond]{\garamond}{EB Garamond}{Maniche}
%\testfont[Carino!]{\garamond}{Adobe Garamond}{Ciao}
\testfont{\libertine}{Linux Libertine}{QuaGlie}

\parte{Transizionali}{Hanno un contrasto pronunciato tra aste verticali e orizzontali. Le grazie sono piatte e gli assi delle lettere sono verticali.}
\testfont[Raccomandato!]{\caslon}{Adobe Caslon}{ColtivaZione}
\testfont{\baskervald}{Baskervald ADF}{Atrofizza}
\testfont[Versione open source di Times New Roman ma con legature]{\termes}{TeX Gyre Termes}{Vegliare}
%\testfont{\troman}{Times New Roman}{Ciao}
\testfont[Figo!]{\sabon}{Sabon LT}{Bustine}
\testfont{\celeste}{CelesteST}{ScodeLla}
\testfont[Versione open source di ITC Bookman]{\bonum}{TeX Gyre Bonum}{Jugoslavi}

\parte{Romani moderni}{Il passaggio tra aste verticali e orizzontali è marcatissimo. Le grazie sono fini e ad angolo retto.}
\testfont{\bodoni}{Bauer Bodoni}{RegolatiVo}
\testfont[Versione open source di New Century Schoolbook]{\schola}{TeX Gyre Schola}{ZigrinAto}
%\testfont{\century}{New Century Schoolbook}{Ciao}
\testfont{\rmfamily}{Computer Modern}{PediCure}
\testfont{\melior}{Melior}{Castori}
\testfont{\raleigh}{Raleigh LT}{VessilLo}

\parte{Grotteschi}{Basso contrasto nel tratto e proporzioni abbastanza regolari. Le forme arrotondate sono spesso ovali.}
\testfont{\grotesque}{Grotesque}{NapolEone}
\testfont[Da risolvere: non mostra virgolette]{\bau}{FF Bau}{XilofOno}

\parte{Neo-grotteschi}{Rispetto ai grotteschi hanno forme più omogenee e arrotondate. I terminali sono orizzontali e le aperture sono strette.}
\testfont{\univers}{Univers}{TemPorali}
\testfont[Versione open source di Helvetica]{\heros}{TeX Gyre Heros}{RiGato}
%\testfont{\helvetica}{Neue Helvetica}{Ciao}

\parte{Gotici senza grazie}{Versione americana dei grotteschi con forme semplici e più statiche. Solitamente hanno un basso contrasto.}
\testfont{\arial}{Arial}{GiRato}
\testfont{\sf}{Latin Modern Sans}{Pubertà}

\parte{Geometrici}{Sono costruiti da forme geometriche quasi sempre circolari o quadrate e riducono al minimo il contrasto del tratto.}
\testfont[Versione open source di ITC Avant Garde Gothic]{\adventor}{TeX Gyre Adventor}{Regionali}
%\testfont{\avantgarde}{ITC Avant Garde Gothic}{Ciao}
\testfont{\din}{FF Din}{Galeoni}
\testfont{\futura}{Futura LT}{HambuRger}

\parte{Umanisti senza grazie}{Hanno una struttura piuttosto calligrafica, con maggiore contrasto nel tratto rispetto ad altri tipi di caratteri senza grazie. Le aperture sono molto larghe.}
\testfont{\gill}{Gill Sans}{Wimbledon}
\testfont[Molto chiaro e leggibile]{\frutiger}{Frutiger}{Mitiga}
\testfont{\myriad}{Myriad}{ForestiEra}
\testfont{\optima}{Optima}{Sequestrai}
\testfont{\verdana}{Verdana}{Ossidriche}

\parte{Neo-umanisti senza grazie}{Simili a quelli umanisti, ma presentano aperture più larghe ed un contrasto del tratto minore.}
\testfont{\dax}{FF Dax}{RaCcoglie}
\testfont{\meta}{FF Meta}{Velocista}
\testfont{\balance}{FF Balance}{Ubriacone}

\parte{Egiziani}{Solitamente aumentano lo spessore delle grazie e delle aste orizzontali. Ricordano lo stile dei caratteri da macchina da scrivere.}
\testfont{\rockwell}{Rockwell}{Far West}
\testfont{\caecilia}{PMN Caecilia}{Estrazione}
\testfont[Versione open source di Courier]{\cursor}{TeX Gyre Cursor}{Computer}

\parte{Scritture}{Tentano di riprodurre un testo scritto a mano. Il loro impiego è limitato per la scarsa leggibilità dei caratteri nei testi lunghi.}
\testfont{\bickham}{Bickham Script}{AltiSsima}
\testfont[Da risolvere: non prende le alternative]{\zapfino}{Zapfino LinoType}{Killer}
\testfont[Versione open source di Zaph Chancery]{\chorus}{TeX Gyre Chorus}{Giostre}

\parte{Medievali}{Richiamano gli antichi caratteri dei monaci amanuensi e delle bibbie di Gutemberg.}
\testfont{\frakfamily}{Fraktur}{Studiosa}
\testfont{\gothfamily}{Gotisch}{RiceTtario}
\testfont{\swabfamily}{Schwabacher}{BerlinEsi}

\end{document}